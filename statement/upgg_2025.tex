\begin{statement}
My research group employs a multidisciplinary strategy to explore how local chromatin environments influence core processes like DNA replication, repair, and transcription\textsuperscript{1,2,3}. We have pioneered novel genomic approaches to study chromatin dynamics and profile chromatin assembly post-replication to elucidate the inheritance of epigenetic information\textsuperscript{4}. Our work melds experimental and computational sciences to decode chromatin-mediated regulation with unprecedented detail.

My paramount responsibility is mentoring a small research group comprised of experimental and computer scientists. Our hands-on training enables my trainees to generate and rigorously interpret extensive genomic data sets, which has fostered their successful careers in both academia and industry.

As Director of Graduate Studies in Pharmacology, I play a pivotal role in promoting the growth of a broad cohort of graduate students with varied backgrounds and research interests. I am responsible for overseeing recruitment, providing academic and professional advice, and ensuring our students feel supported and motivated throughout their graduate journey. My dedication to developing scientific talent extends beyond my laboratory and the pharmacology program; I am a faculty mentor for Duke's BioCORE program and I also serve on the selection committee for the Dean's Graduate Fellowships which aim to identify and support exceptional students from a wide range of academic backgrounds across the entire university.

My commitment to education and the University Program in Genetics and Genomics extends into the classroom where I teach or have taught various modules, including a specialized course on computational biology for experimental biologists, a module on chromatin assembly and epigenetic inheritance, and most recently, a module on using Large Language Models (LLMs) in biomedical research. These courses provide essential training in rigor and reproducible research methodologies, preparing students to conduct responsible, reliable, and high-quality scientific investigations across diverse genomic disciplines.

Additionally, I am an active participant in the research community. I serve on the editorial board of \textit{Genome Research} and frequently review grants for the American Cancer Society and the NIH.  

\begin{enumerate}

\item Hoffman RA, MacAlpine HK, \textbf{MacAlpine DM}. Disruption of origin chromatin structure by helicase activation in the absence of DNA replication. Genes Dev. 2021 Oct 1;35(19-20):1339-1355. doi: 10.1101/gad.34857.121. Epub 2021 Sep 23. PMC8494203.

\item Tripuraneni V, Memisoglu G, MacAlpine HK, Tran TQ, Zhu W, Hartemink AJ, Haber JE, \textbf{MacAlpine DM}. Local nucleosome dynamics and eviction following a double-strand break are reversible by NHEJ-mediated repair in the absence of DNA replication. Genome Res. 2021 May;31(5):775-788. doi: 10.1101/gr.271155.120. Epub 2021 Apr 2. PMC8092003.

\item Tran TQ, MacAlpine HK, Tripuraneni V, Mitra S, \textbf{MacAlpine DM}, Hartemink AJ. Linking the dynamics of chromatin occupancy and transcription with predictive models. Genome Res. 2021 Jun;31(6):1035-1046. doi: 10.1101/gr.267237.120. Epub 2021 Apr 23. PMC8168580. * Co-corresponding author

%\item Gutiérrez MP, MacAlpine HK, \textbf{MacAlpine DM}. Nascent chromatin occupancy profiling reveals locus- and factor-specific chromatin maturation dynamics behind the DNA replication fork. Genome Res. 2019 Jul;29(7): 1123-1133. doi: 10.1101/gr.243386.118. Epub 2019 Jun 19. PMC6633257.

\item Chen B, MacAlpine HK, Hartemink AJ, \textbf{MacAlpine DM}. Spatiotemporal kinetics of CAF-1-dependent chromatin maturation ensures transcription fidelity during S-phase. Genome Res. 2023 Dec 11;33(12):2108–18. doi: 10.1101/gr.278273.123. PMC10760526.
\end{enumerate}

\end{statement}

