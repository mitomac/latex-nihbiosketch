\begin{statement}
My research group employs a multidisciplinary strategy to explore how local chromatin environments influence core processes like DNA replication\textsuperscript{2}, repair\textsuperscript{2}, and transcription\textsuperscript{3}. We have pioneered novel genomic approaches to study chromatin dynamics and profile chromatin assembly post-replication to elucidate the inheritance of epigenetic information\textsuperscript{5}. Our work melds experimental and computational sciences to decode chromatin-mediated mechanisms that ensure genetic and epigenetic inheritance.  By elucidating how chromatin assembly post-replication contributes to the inheritance of epigenetic information, our work directly addresses the mechanisms through which cells maintain or disrupt genetic integrity. This is particularly relevant in cancer, where the epigenetic landscape can influence tumor suppressor genes and oncogenes, affecting cell proliferation, apoptosis, and metastasis. Our multidisciplinary approach not only sheds light on fundamental biological processes but also opens avenues for developing targeted therapies that could reverse or modulate epigenetic changes associated with cancer. 

As a member of the DCI, I contribute to a number of groups focused on mechanisms of genome stability.  I am also grateful for the shared resources provided by the DCI which strengthen and augment my research program.   

In addition to my research program, I am committed to the educational mission of the University.   I teach a number of courses, including a specialized course on computational biology for experimental biologists. This course provides essential training in rigor and reproducible research involving large genomic datasets, preparing students to conduct responsible, reproducible, and high-quality research.

 Finally, I am an active participant in the research community. I serve on the editorial board of \textit{Genome Research} and frequently review grants for the American Cancer Society, National Science Foundation, and the National Institutes of Health.  

\begin{enumerate}

\item Hoffman RA, MacAlpine HK, \textbf{MacAlpine DM}. Disruption of origin chromatin structure by helicase activation in the absence of DNA replication. Genes Dev. 2021 Oct 1;35(19-20):1339-1355. doi: 10.1101/gad.34857.121. Epub 2021 Sep 23. PMC8494203.

\item Tripuraneni V, Memisoglu G, MacAlpine HK, Tran TQ, Zhu W, Hartemink AJ, Haber JE, \textbf{MacAlpine DM}. Local nucleosome dynamics and eviction following a double-strand break are reversible by NHEJ-mediated repair in the absence of DNA replication. Genome Res. 2021 May;31(5):775-788. doi: 10.1101/gr.271155.120. Epub 2021 Apr 2. PMC8092003.

\item Tran TQ, MacAlpine HK, Tripuraneni V, Mitra S, \textbf{MacAlpine DM}*, Hartemink AJ. Linking the dynamics of chromatin occupancy and transcription with predictive models. Genome Res. 2021 Jun;31(6):1035-1046. doi: 10.1101/gr.267237.120. Epub 2021 Apr 23. PMC8168580. *Co-corresponding author

%\item Gutiérrez MP, MacAlpine HK, \textbf{MacAlpine DM}. Nascent chromatin occupancy profiling reveals locus- and factor-specific chromatin maturation dynamics behind the DNA replication fork. Genome Res. 2019 Jul;29(7): 1123-1133. doi: 10.1101/gr.243386.118. Epub 2019 Jun 19. PMC6633257.

\item Chen B, MacAlpine HK, Hartemink AJ, \textbf{MacAlpine DM}. Spatiotemporal kinetics of CAF-1-dependent chromatin maturation ensures transcription fidelity during S-phase. Genome Res. 2023 Dec 11. doi: 10.1101/ gr.278273.123. Epub ahead of print. PMC In Process.

\end{enumerate}

\end{statement}