\begin{statement}
As a Professor of Pharmacology and Cancer Biology, my research group at Duke University adopts a multidisciplinary approach, intersecting genomic studies with advanced computational methods. We've developed innovative genomic techniques to unravel chromatin dynamics, focusing on how local chromatin environments impact DNA replication\textsuperscript{1}, repair\textsuperscript{2}, and transcription\textsuperscript{3}. This work is critical for understanding the inheritance of epigenetic information and the underlying mechanisms ensuring genetic stability.

My laboratory's strength lies in blending experimental science with computational analysis. I am adept in cloud computing, utilizing platforms such as Azure, AWS, and GCP, which has been instrumental in our research. 
 I was also a co-PI for NHGRI's modENCODE/ENCODE Data Analysis Center and have experience analyzing and integrating diverse genomic datasets to derive high impact discoveries\textsuperscript{4}.  
 
\noindent I take pride in my commitment to open science. Our public GitHub repository ensures full transparency and reproducibility of our research. This repository includes data and code for the majority of our publications, allowing peers  to replicate and build upon our work.

\noindent In my educational endeavors, I focus on bridging gaps between experimental biology and computational analysis. I teach a specialize course in computational biology for experimental biologists, emphasizing rigorous and reproducible research with large genomic datasets.  I have also pioneered the use of Jupyter notebooks in the classroom, integrating them with virtual Docker containers managed from Duke's private GitHub/GitLab repository. This innovative approach, developed in collaboration with Duke's Office of Information Technology, enhances hands-on learning and has been well-received by students with little to no prior training in computational biology.

\noindent As Director of Graduate Studies in Pharmacology, I am deeply involved in nurturing a diverse and vibrant community of graduate students. My role encompasses recruitment, mentorship, and academic guidance, ensuring a supportive and enriching environment for their professional growth. Additionally, my work extends to promoting diversity university-wide, as a faculty mentor for Duke's BioCORE program and a member of the selection committee for the Dean's Graduate Fellowships.

\noindent Beyond my institutional roles, I contribute actively to the broader research community. I am on the editorial board of \textit{Genome Research} and regularly review grants for the American Cancer Society, National Science Foundation, and the National Institutes of Health.

\noindent My overarching goal is to continue fostering an environment of collaborative learning, innovative research, and dedication to the advancement of science and education.


% \noindent My research group employs a multidisciplinary strategy to explore how local chromatin environments influence core processes like DNA replication\textsuperscript{1}, repair\textsuperscript{2}, and transcription\textsuperscript{3}. We have pioneered novel genomic approaches to study chromatin dynamics and profile chromatin assembly post-replication to elucidate the inheritance of epigenetic information\textsuperscript{4}. Our work melds experimental and computational sciences to decode chromatin-mediated mechanisms that ensure genetic and epigenetic inheritance. My paramount training responsibility is mentoring a small research group comprised of both experimental and computer scientists. Our hands-on training enables my trainees to generate and rigorously interpret extensive genomic data sets, which has fostered their successful careers in both academia and industry.

% \noindent As Director of Graduate Studies in Pharmacology, I play a pivotal role in promoting the growth of a diverse cohort of graduate students. I am responsible for overseeing recruitment, providing academic and professional advice, and ensuring our students feel supported and motivated throughout their graduate journey. My dedication to fostering diversity extends beyond my laboratory and the pharmacology program; I am a faculty mentor for Duke's  BioCORE program and I also serve on the selection committee for the Dean's Graduate Fellowships which are aimed at promoting diversity across the entire university.  

% \noindent My commitment to education extends into the classroom where I teach various courses, including a specialized course on computational biology for experimental biologists. This course provides essential training in rigor and reproducible research involving large genomic datasets, preparing students to conduct responsible, reproducible, and high-quality research.

% \noindent Finally, I am an active participant in the research community. I serve on the editorial board of \textit{Genome Research} and frequently review grants for the American Cancer Society, National Science Foundation, and the National Institutes of Health.  

\begin{enumerate}

\item Hoffman RA, MacAlpine HK, \textbf{MacAlpine DM}. Disruption of origin chromatin structure by helicase activation in the absence of DNA replication. Genes Dev. 2021 Oct 1;35(19-20):1339-1355. doi: 10.1101/gad.34857.121. Epub 2021 Sep 23. PMC8494203.

\item Tripuraneni V, Memisoglu G, MacAlpine HK, Tran TQ, Zhu W, Hartemink AJ, Haber JE, \textbf{MacAlpine DM}. Local nucleosome dynamics and eviction following a double-strand break are reversible by NHEJ-mediated repair in the absence of DNA replication. Genome Res. 2021 May;31(5):775-788. doi: 10.1101/gr.271155.120. Epub 2021 Apr 2. PMC8092003.

\item Tran TQ, MacAlpine HK, Tripuraneni V, Mitra S, \textbf{MacAlpine DM}*, Hartemink AJ. Linking the dynamics of chromatin occupancy and transcription with predictive models. Genome Res. 2021 Jun;31(6):1035-1046. doi: 10.1101/gr.267237.120. Epub 2021 Apr 23. PMC8168580. *Co-corresponding author

\item Ho JW, Jung YL, Liu T, Alver BH, Lee S, Ikegami K, Sohn KA, Minoda A, Tolstorukov MY, Appert A, Parker SC, Gu T, Kundaje A, Riddle NC, Bishop E, Egelhofer TA, Hu SS, Alekseyenko AA, Rechtsteiner A, Asker D, Belsky JA, Bowman SK, Chen QB, Chen RA, Day DS, Dong Y, Dose AC, Duan X, Epstein CB, Ercan S, Feingold EA, Ferrari F, Garrigues JM, Gehlenborg N, Good PJ, Haseley P, He D, Herrmann M, Hoffman MM, Jeffers TE, Kharchenko PV, Kolasinska-Zwierz P, Kotwaliwale CV, Kumar N, Langley SA, Larschan EN, Latorre I, Libbrecht MW, Lin X, Park R, Pazin MJ, Pham HN, Plachetka A, Qin B, Schwartz YB, Shoresh N, Stempor P, Vielle A, Wang C, Whittle CM, Xue H, Kingston RE, Kim JH, Bernstein BE, Dernburg AF, Pirrotta V, Kuroda MI, Noble WS, Tullius TD, Kellis M, \textbf{MacAlpine DM}*, Strome S, Elgin SC, Liu XS, Lieb JD, Ahringer J, Karpen GH, Park PJ. Comparative analysis of metazoan chromatin organization. Nature. 2014 Aug 28;512(7515):449-52. PMC4227084. * Co-corresponding author
%\item Gutiérrez MP, MacAlpine HK, \textbf{MacAlpine DM}. Nascent chromatin occupancy profiling reveals locus- and factor-specific chromatin maturation dynamics behind the DNA replication fork. Genome Res. 2019 Jul;29(7): 1123-1133. doi: 10.1101/gr.243386.118. Epub 2019 Jun 19. PMC6633257.

%\item Chen B, MacAlpine HK, Hartemink AJ, \textbf{MacAlpine DM}. Spatiotemporal kinetics of CAF-1-dependent chromatin maturation ensures transcription fidelity during S-phase. Genome Res. 2023 Dec 11. doi: 10.1101/ gr.278273.123. Epub ahead of print. PMC In Process.

\end{enumerate}

\end{statement}